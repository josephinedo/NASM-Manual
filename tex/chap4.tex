\chapter{Functions and Macros}
Procedures or subroutines are very important in assembly language, as the assembly language programs tend to be large in size \cite{website:NASM-TutorialsPoint}. In this chapter we are going to present the use of functions (subroutines) along with the passing of parameters in C and we will finish with the explanation of macros.

\section{Functions}
A subroutine is a unit of self-contained code, designed to carry out a specific task and has a determining role in the development of programs in a structured manner.

An assembler subroutine would be equivalent to a function in C. In this section we will see how to define assembler subroutines and how we can use them later from a C program. First we will describe how to work with assembler subroutines:

\begin{itemize}
\item Definition of assembler subroutines.
\item Call and return of subroutine.
\item Passing of parameters to the subroutine and return of results.
\end{itemize}

Next we will see how to make calls to subroutines made in assembler from a program in C and what implications it has in the passing of parameters.

\subsection{Subroutines}
Basically, a subroutine is a set of instructions that start its execution in a point of code identified with a label that will be the name of the subroutine, and ends with the execution of a \texttt{ret} instruction, subroutine return instruction, which causes a jump to the next instruction from where the call was made (call).

Important considerations when defining a subroutine:
\begin{itemize}
\item We must store the modified records within the subroutine to leave them in the same state they were in at the time the call was made to the subroutine, except for the records that are used to return a value. To store the modified records we will use the stack.
\item To maintain the structure of a subroutine and for the program to work properly, you can not perform jumps to instructions of the subroutine; we will always finish the execution of the subroutine with the instruction ret.
\end{itemize}
\begin{minted}{nasm}
subrutine:
	; Save on the stack
	; the modified registers inside of the routine
	;
	; Routine instructions
	;
	; Restore the state of the modified registers
	; Recover the initial value stored on the stack
	ret
\end{minted}

\subsubsection{Passing of parameters}
A subroutine may need to be transferred parameters; the parameters can be passed through registers or the stack. The same happens with the return of results, which can be done by means of registration or the stack. We will consider the cases in which the number of input and return parameters of a subroutine is fixed.

As a working example we will show the use of the conditional statements and the way to define subroutines using the next example to calculate the greatest common divisor (Source Code \ref{listing:gcd}).

\newpage
\begin{listing}[H]
\inputminted{nasm}{code/gcd.asm}
\caption{Conditional statements. gcd.asm}
\label{listing:gcd}
\end{listing}
\newpage

\subsubsection{Calling assembly functions from C}
To use assembler functions within a C program we must define them in the C program, but without implementing them; only the header of the functions is included. Headers include the return type of the function, the name of the function, and the data types of each function parameter.

Once the functions have been defined, they can be called like any other program function. To show this we are going to assembly two codes: gcd.asm (Source Code \ref{listing:gcd}) and factorial.asm (Source Code \ref{listing:factorial}).

\begin{listing}[H]
\inputminted{nasm}{code/factorial.asm}
\caption{Cyclic statements. factorial.asm}
\label{listing:factorial}
\end{listing}

Now we can write a short code in C to use our assembly codes gcd.asm and factorial.asm, but first we need to assembly the codes.

\begin{minted}{bash}
	nasm -fmacho64 gcd.asm
	nasm -fmacho64 factorial.asm
\end{minted}

This will generate the object files gcd.o and factorial.o and then we can link this with a C program for the passing of arguments (Source Code \ref{listing:menu}).

\begin{minted}{bash}
	gcc gcd.o factorial.o menu.c && ./a.out
\end{minted}

\newpage
\begin{listing}[H]
\inputminted{c}{code/menu.c}
\caption{Passing arguments. menu.c}
\label{listing:menu}
\end{listing}

\section{Macros}
Writing a macro is another way of ensuring modular programming in assembly language.
\begin{itemize}
\item A macro is a sequence of instructions, assigned by a name and could be used anywhere in the program.
\item In NASM, macros are defined with \%macro and \%endmacro directives.
\item The macro begins with the \%macro directive and ends with the \%endmacro directive.
\end{itemize}
The syntax fro the macro definition is like follows:
\begin{minted}{nasm}
%macro macroName  numberParams
	; macro body
%endmacro
\end{minted}
Where, numberParams specifies the number parameters, macroName specifies the name of the macro. The macro is invoked by using the macro name along with the necessary parameters. When you need to use some sequence of instructions many times in a program, you can put those instructions in a macro and use it instead of writing the instructions all the time.

For example, a very common need for programs is to write a string of characters in the screen. For displaying a string of characters, you need the following sequence of instructions

\begin{minted}{nasm}
	mov    edx, len	     ;message length
	mov    ecx, msg 	    ;message to write
	mov    ebx, 1        ;file descriptor (stdout)
	mov    eax, 4        ;system call number (sys_write)
	int    0x80          ;call kernel
\end{minted}

In the above example of displaying a character string, the registers EAX, EBX, ECX and EDX have been used by the INT 80H function call. So, each time you need to display on screen, you need to save these registers on the stack, invoke INT 80H and then restore the original value of the registers from the stack. So, it could be useful to write two macros for saving and restoring data.

We have observed that, some instructions like IMUL, IDIV, INT, etc., need some of the information to be stored in some particular registers and even return values in some specific register(s). If the program was already using those registers for keeping important data, then the existing data from these registers should be saved in the stack and restored after the instruction is executed (Source Code \ref{listing:macro}).

\newpage
\begin{listing}[H]
\inputminted{nasm}{code/macro.asm}
\caption{Macro example. macro.c}
\label{listing:macro}
\end{listing}
