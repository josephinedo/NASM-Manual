\chapter{Programming Concepts in Assembly}
In this chapter we will see how the different control structures of the C language can be translated into assembler language.

\section{Conditional Statement IF}
In C programming we have:
\begin{minted}{c}
if (a > b) {
	maxA = 1;
	maxA = 0;
}
\end{minted}
We can translate this code in assembly as follows:
\begin{minted}{nasm}
	mov rax, qword [a]    ; Loads 'a' and 'b' into registers rax and rbx
	mov rbx, qword [b]
	cmp rax, rbx          ; The comparison is made
	jg L1                 ; If condition is true jumps to L1
	jmp L2                ; If condition is false jumps to L2
L1:
    mov byte [maxA], 1    ; Only executed if condition was true
    mov byte [maxB], 0	
L2:
\end{minted}

\section{Conditional Statement IF-ELSE}
In C programming we may have:
\begin{minted}{c}
if (a > b) {
	max = a;
} else {
	max = b;
}
\end{minted}
We can translate this code in assembly as follows:
\begin{minted}{nasm}
	mov rax, qword [a]    ; loads 'a' and 'b' to registers rax and rbx
	mov rbx, qword [b]
	cmp rax, rbx          ; The comparison is made
	jg L1                 ; If condition is true, jumps to L1
	mov byte [max], 'b'   ; else (a <= b)
	jmp L2
L1:
	mov byte [max], 'a'   ; if (a > b)
L2:
\end{minted}

\section{Conditional Statement WHILE}
In C programming we have the next factorial example:
\begin{minted}{c}
result=1;
while (num > 1){
    result = result * num;
	num--; 
}
\end{minted}
which can be written in assembly as follows:
\begin{minted}{nasm}
	mov rax, 1               ; rax its going to be the result
while:
	cmp qword [num], 1       ; The comparison is made
	jle L1
	imul rax, qword [num]    ; rax=rax*[num]
	dec qword [num]
	jmp while
L1:
	mov qword [result], rax
\end{minted}

\section{Conditional Statement DO-WHILE}
Another example in C...
\begin{minted}{c}
result=1;
do {
	result = result * num;
	num--;
} while (num > 1)
\end{minted}
The difference with the WHILE structure is subtle
\begin{minted}{nasm}
	mov rax, 1            ; rax its going to be [result]
	mov rbx, qword [num]  ; Loads num in rbx
while:
	imul rax, rbx
	dec rbx
	cmp rbx, 1            ; the comparison is made
	jg while              ; It it is true then jumps tu while
	mov qword [result], rax
	mov qword [num], rbx
\end{minted}

\section{Conditional Statement FOR}
Finally the FOR structure example in C
\begin{minted}{c}
result=1;
for (i = num; i >1; i--) {
	result = result * i;
}
\end{minted}
The difference with the WHILE structure is subtle
\begin{minted}{nasm}
	mov rax, 1            ; rax its going to be [result]
	mov rcx, qword [num] ; rcx [i] initialized with [num]
for:
	cmp rcx, 1            ; The comparison is made
	jg cierto             ; If true jumps to L1
	jmp L2
L1:
	imul rax,rcx
	dec rcx
	jmp for
L2:
	mov qword [result], rax
	mov qword [i], rcx
\end{minted}



