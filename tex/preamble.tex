\documentclass[letterpaper,11 pt]{report}

%***************************************** ENCODING ********************************************
\usepackage[utf8]{inputenc} % UTF-8 -input- enconding, useful to put á instead of \'a
\usepackage[T1]{fontenc} % T1 -font- encoding, with this PDF viewer can look for my special words
%\usepackage[spanish,mexico]{babel} % My language in the text

%***************************************** GEOMETRY ********************************************
\usepackage[pass]{geometry}
%\linespread{1.3} To make line spacing
\usepackage{comment}
\usepackage{enumitem} % Remove indent from itemize and list environment

%***************************************** MATH ************************************************
\usepackage{amsmath} 
\usepackage{amsfonts}	% AMS packages
\usepackage{amssymb}
\numberwithin{equation}{chapter} % To put equation number in a different way 
\usepackage[alsoload=synchem]{siunitx} % For units and degree symbol with \ang{ }

%***************************************** TOC  *************************************************
\usepackage[nottoc]{tocbibind} 		% Put figures, tables, bibliography and index in TOC
\setcounter{secnumdepth}{3}
\setcounter{tocdepth}{3} % This define the deep of the chapter-section-subsection-etc in TOC

%***************************************** COLOR  ***********************************************
\usepackage[usenames,dvipsnames,table,svgnames]{xcolor} 	% Use of color in different environments

%***************************************** FIGURES  **********************************************
\usepackage{graphicx} 	% eps figures using PDFLatex
\usepackage{wrapfig}
\usepackage{subcaption} % To multiple figures
\usepackage{pdflscape} % To landscape figures
\usepackage{rotating} % To rotate figures
\graphicspath{{fig/}}

%***************************************** TABLES  **********************************************
\usepackage{tabularx} 		% To rezise the table to textwith
\newcolumntype{C}{>{\centering\arraybackslash}X}	 % Definition needed to center the text in the tables

%***************************************** CAPTION  *********************************************
\usepackage{caption} % Put beautiful captions on tables and figures
\captionsetup{font={small},labelfont=bf, labelsep=endash,format=plain,indention=.5cm,margin={1cm,1cm}}

%***************************************** TITLESEC  *********************************************
\usepackage{fix-cm} % To put arbirary font sizes, used on chapter style
\usepackage{titlesec} % To change the style of chapters, sections, subsections, etc. See documentation of titlesec for more information.

\titleformat{\chapter}[display]
	{\bfseries\huge\scshape}
	{\color{BrickRed}\fontsize{60}{80}\selectfont\thechapter}
  	{2ex}{}[]
\titlespacing{\chapter}{4pc}{*0}{*10}[0pc]

\titleformat{\section}
	{\bfseries\large\scshape}
  	{\color{BrickRed}\thesection}{0.3 em}{}
\titlespacing{\section}{-2pc}{*4}{*1}[0pc]

\titleformat{\subsection}
	{\bfseries\large\scshape}
  	{\thesubsection}{0.3 em}{}
\titlespacing{\subsection}{-2pc}{*4}{*1}[0pc]

%***************************************** FANCYHDR ********************************************
\usepackage{fancyhdr} % Used to change the style of the headers and footers
\setlength{\headheight}{15 pt}
\pagestyle{fancy}

\renewcommand{\chaptermark}[1]{ \markboth{\MakeUppercase{#1}}{} }
\renewcommand{\sectionmark}[1]{ \markright{\MakeUppercase{#1}}{} }
\renewcommand{\headrulewidth}{0pt} % To remove the line in the header

\lhead[{\Large\textbf{\thepage}}\hspace{1cm}\leftmark]{}
\rhead[]{\rightmark\hspace{1cm}{\Large\textbf{\thepage}}}
\fancyfoot[OC]{ } % I put this to erase the page numbers from the bottom
\fancyheadoffset{2 cm}  % To make the header larger

%***************************************** EMPTY PAGE ******************************************
\makeatletter
\let\ps@plain\ps@empty  % This redifine the pagestyle{plain} to pagestyle{empty}, that means no numberpage in the first page of chapters, TOC, etc
\makeatother

%***************************************** NOMENCLATURE **************************************
%\usepackage{nomencl}
%\makenomenclature
%\renewcommand{\nomname}{Nomenclatura}

%***************************************** CODE *************************************************
\usepackage{listings} % This is to put source code (C, C++, Python, etc) in the PDF
\usepackage{minted} % Other tool to put code in latex
\setminted[nasm]{tabsize=4, style=autumn, bgcolor=FloralWhite, fontsize=\footnotesize}
\setminted[c]{tabsize=4, style=autumn, bgcolor=AliceBlue, fontsize=\footnotesize}
\usepackage{verbatim} % This is to put wherever it is in the TEX source code
\renewcommand\listoflistingscaption{List of source codes}
\renewcommand\listingscaption{Source Code}

%***************************************** DATE *************************************************
\usepackage{datetime} % This is to change \today format
\newdateformat{mydate}{\monthname[\THEMONTH], \THEYEAR}

%***************************************** BIBLIO ************************************************
\bibliographystyle{unsrt} % Put some style in the bibliography

%***************************************** INDEX ************************************************
\usepackage{makeidx} % To make the index
\makeindex

%***************************************** LINKS ************************************************
\usepackage[hidelinks]{hyperref} % To make some links in the PDF, this package should be at the end